% !TeX root = ../main.tex
% Add the above to each chapter to make compiling the PDF easier in some editors.

\chapter{Introduction}\label{chap:introduction}

Developing a website has become very easy in recent years. A basic model of a personal site can be made with
a decent amount of work and an HTML template. Even such complicated software as a whole operation system
(Linux for example) is open-sourced and available to everybody with an internet connection.
Linux is even the most used operating system for web servers worldwide \cite{w3techs_usage_op_website}.
Developing a CPU on the other side is an entirely different story. Almost all CPU architectures are
closed source and owned by large companies. The execution of most modern CPUs cannot be tracked, in
fact, many contain a so-called secure zone or trusted zone that handles DRM content and security-related
work like storing cryptography keys \cite{microcontrollertips_arm_trustzone_explained}.
A general distrust against these proprietary zones has risen among enthusiast users that motivated
some users so far that they researched the Intel Active-Management-Technology (AMT) and found a security
vulnerability by which they could completely disable the functionality \cite{github_disable_intel_amt}.
However, these more or less official hidden layers are of small concern when we consider the thread of
undocumented backdoors built into the hardware. Modern x86 processors contain hidden subprocessors
that might have unrestricted access to the execution of the main processor. One security researcher
found such a backdoor and used it to gain root access on an old VIA x86 system 
\cite{christopher_domas_god_mode_unlocked}.
Furthermore, the undocumented features are not the only problem of the current generation of the processor: in January
of 2018, two new security vulnerabilities surfaced that deeply shattered the security of many processors.
Spectre \cite{Kocher2018spectre} hit AMD processors, Intel processors and many other chips, that use 
out-of-order execution and Meltdown \cite{Lipp2018meltdown} hit manly Intel processors with speculative
execution (starting from 1995, nearly all processors). These flaws, for the 
most part can be fixed with an update,
but the current implementation of the fixes lowers the performance of the system and are thus not ideal.
New hardware mitigations have to be implemented.
Apart from the user perspective, the situation for producers of electronic components is not ideal either.
Many manufacturers need processors to power their products, of which a prominent example is Nvidia. Graphics cards
have become very complex and thus also need co-processors. Developing a new processor for every new product
is not feasible for these companies. Currently, they need to buy processors from vendors. The problem with this
approach is that the bought architecture often cannot be easily extended with a needed extension or
an accelerator which a new product needs let alone it is even possible to extend it. These bought
custom solutions are not cheap either.

Many companies and associations have tried to solve the issues of developing a flexible and open ISA, 
but no other has reached as far as the Berkeley University of California with their RISC-V design.
It is designed to be easily extendable and said to fit microcontrollers, as well as to many core
massive computing processors like the SuperMUC NG in Garching uses.
Safety concerns can also be addressed if the code of a RISC-V implementation is published
and audited. No Spectre or Meltdown vulnerability has yet been seen in a RISC-V implementation
although this statement heavily relies on the correctness of the implementation.
This bachelor thesis will dive deeper into the extendability of the RISC-V design and the
implementation of the design from the Berkeley University.
With the new possibility of extending a CPU design, we want to look into the probability
of implementing a Checkpoint/Restore mechanism in hardware.