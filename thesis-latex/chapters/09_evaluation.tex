\chapter{Evaluation}\label{chap:evaluation}
In the course of this bachelor thesis, we explored how the
RISC-V ISA is set up and how the implementation of the
University of Berkeley rocket-chip can be extensively
configured to fit many use-cases.
The start of the bachelor thesis was mainly dominated
by the setup of the Xilinx boards the included Xilinx Vivado
development suite.
Sadly the fpga-zynq repository is no longer up to date, and thus
the attempt to bring a working Rocket-core on an FPGA-ARM board did
not succeed. Even after combined efforts with colleagues from the
chair of operating systems could not get the Linux image to boot
on the Xilinx boards.
The experience with the Xilinx Arty board in combination
with the up-to-date Freedom repository from SiFive was better but
not still not perfect. A guide has been written, which should make
the setup of the E300 platform on the Xilinx Arty board easier.
After the setup up of the board, it was discovered, that it does contain
a large enough FPGA to support a Rocket Chip MMU.
Nonetheless, the gained knowledge of working with the Freedom
platform can still be used for later attempts of bringing
the larger SiFive U500 platform on the new Xilinx VC707 FPGA board
at the chair. The included quad-core in the U500 platform
contains an MMU and is capable of running Linux.
For further research, the included emulator was used.
The following phase in
section \ref{sec:rocket-chip} was dominated by the studies of
the Rocket Chip project and the structure of the Rocket core.
As the concept of the hardware construction language Chisel
is new and the Rocket Chip project contains only a small number
of comments the initial barrier to understanding the structure was high.
With the help of the people from the RISC-V and BOOM mailing list,
initial questions could be resolved, and the system was easier to understand.
After further analysis of the Rocket Core, one found the
caching structure to be more complicated than previously
anticipated but well thought out.
The scope of the implementation will need to include some
change to the translation lookaside buffers, located in the
L1 caches of the Rocket tiles.
The concept in chapter \ref{chap:concept} could not be
implemented, as time was restricted.

For future approaches, it might be beneficial to start first
with a virtual emulator instead of a physical FPGA attempt,
to speed up the initial discovery process of the Rocket-Chip
generator.