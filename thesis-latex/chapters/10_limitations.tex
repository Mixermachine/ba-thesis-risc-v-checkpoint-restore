\chapter{Limitations}
\label{chap:limitations}
\section{Xilinx boards and Vivado}
In the course of this thesis many problems were encountered.
The first problem that has been met was the deprecation of the
fpga-zynq repository, which made the attempt to bring
a working RISC-V core on the Xilinx boards Zybo Zynq-7000
and Zedboard fail.
The compatibility of the Xilinx Vivado development suite sometimes
seems to be unclear. The SiFive Freedom repository, for example,
recommends the version 2017.1 or newer for the E300 platform as
"old versions are known to fail". The recommended version for the
U500 platform, on the other hand, is 2016.04 or older as
"newer versions are known to fail"
\footnote{https://github.com/sifive/freedom}.

Also, in the course of this Bachelor thesis, a patch update from
2018.2 to 2018.2.2 provoked a failure in the compilation process.
Reverting to the previous version fixed the issue.

\section{Rocket-chip repository}
As we have seen in the chapter \ref{chap:implementation} the
Rocket-chip project file \textit{PTW.scala} did not contain many lines of
comments. After some further inspection of the Rocket-chip repository
and the use of the Scalastyle inspection \footnote{http://www.scalastyle.org/}
in IntelliJ it is apparent, that the project contains a meager
number of comments. This makes it hard for beginners to understand
the logic of the project. One developer in the mailing list recommended
to execute the virtual tests and analyze the resulting waveform file as the
good way to understand the logic
\footnote{\url{https://groups.google.com/forum/?utm_medium=email&utm_source=footer\#!msg/riscv-boom/jZ1gC4Qs35Q/8HltWll_CgAJ}}.

\section{Concept}
The presented concept mainly contains two limitations.
First, there is the reduction of addressable virtual memory, as the
proposed method splits the virtual memory space into \textit{n} equally sized
parts. This can limit larger applications, especially if many checkpoint
slots are necessary.
The second limitation is the performance penalty for recursively searching
the current working copy of a page table entry, as every new incremental
checkpoint adds a new page table entry.
The last limitation is the higher memory requirement for the TLB, which
first needs to hold two more bits per PTE and secondly has to store
n PTEs in the worst case to access one working copy for one memory page.