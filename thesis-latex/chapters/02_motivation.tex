\chapter{Motivation}\label{chap:motivation}
The chair of operating systems at the Technical University Munich is currently 
refining an operating system called Genode that should be fault tolerant
and hardened against attackers.
The system should be able to support real-time applications in the
field of car electronics. One essential part of making this system
fault tolerant is to implement the technique of Checkpoint/Restore for
the RAM system. This technique makes it possible to manage a complete failure of
one unit with little to no noticeable downtime and small additional
computational expenses as long as a mirror unit is available.
The previous implementations of the Checkpoint/Restore technique have
been implemented in software and thus are slowing the system down,
as fast access to memory is crucial for modern computer
systems \cite{sebastian_bachmaier_optimizing_genode_cr_fpga}.
When searching for new ways to implement a hardware solution
for this technique, one soon comes across the RISC-V architecture.
This instruction set architecture (ISA) is completely open source
and can be modified by everybody. It is built to be modular and expandable.
Major big companies like Nvidia \cite{risc-v_member_nvidia} and 
Western Digital \cite{risc-v_member_wd}
joined the RISC-V consortium and want to use the new ISA extensively.
Western Digital as a Platinum member has already developed and
open sourced a RISC-V
instruction set simulator called Whisper \cite{wd_github_whisper}, \cite{github_swerv_iss} and a
RISC-V implementation called SweRV Core \cite{wd_swerv_core}, \cite{github_swerv_core}.
Western Digital ships more than one billion processors each year
and plans to migrate the ARM-based processors to RISC-V
\cite{risc-v_news_wd_one_billion_chips}.
With the new possibility of directly customizing a processor new
ways of implementing a Checkpoint/Restore mechanism can be found,
that do not suffer from drastic performance reductions.
Even though custom hardware modules exist, a direct implementation
in the CPU is beneficial, as it can be faster.

Taking the traction the RISC-V project has gained in recent years
this topic looks very promising.