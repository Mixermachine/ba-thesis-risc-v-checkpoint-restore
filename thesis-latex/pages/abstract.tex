\chapter{\abstractname}

Modern processors are used in nearly every application today,
and they are getting more and more important. User depend increasingly
on the correct function of modern machines. The software of a
self-driving car, for example, is a very critical application that should
not completely crash. The same holds for modern medical devices
and other mission-critical appliances.
To reduce the impact of hardware failures and other unforeseen events
systems are build with redundant hardware. In the case of failure,
the redundant part of the system can take over.
Depending on the system this step can take valuable time, as the
system needs to initialize and boot up to the state where the
previous system has failed.
Checkpoint/Restore mechanisms are put in place to reduce the impact
of such defects further, as a checkpointed system only loses the
progress since the last checkpoint in case of a crash.

As most of the current Checkpoint/Restore mechanisms are implemented
in software, they introduce certain bottlenecks into the system.
Especially the tracking of changes to checkpointed memory and the copying
of such memory to conserve previous states has proven to be costly.
The most currently available hardware merely is not made with
Checkpoint/Restore systems in mind.
To accelerate critical systems hardware acceleration have since been used.
Hardware modules that are connected over the system bus or other connections
might not be fast enough for high-speed systems. Thus this thesis
will look in the possibility to directly modify components in the CPU
to accelerate Checkpoint/Restore algorithms further.
To freely modify a CPU, one needs an open CPU architecture.
In this thesis, the upcoming RISC-V instruction set architecture was chosen
as it looks promising and can be freely obtained. The RISC-V ISA is a
free blueprint of a processor architecture and was initially developed
by the University of California, Berkeley. As for the implementation
part of this thesis, we will use three kinds of Xilinx FPGA boards, the included
development suite Xilinx Vivado and the RISC-V implementation
Rocket Chip as it appears to be the most popular RISC-V project as of the time of
this thesis.
Later this thesis will present a Checkpoint/Restore mechanism that
builds on the modification of the memory management unit (MMU) in the CPU.
We will explore how the Rocket Chip project has integrated
the MMU in their Rocket cores and how we could apply the proposed concept.
Due to time constraints, the concept could not be implemented.